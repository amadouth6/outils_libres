\documentclass[a4paper]{article}
\usepackage[utf8]{inputenc}
\usepackage{graphicx}
\graphicspath{ {/home/amadou/Bureau/IUT/Outils_Libres_Cote_client/Capture/} }


\title{Rapport Outils Libres} 

\author{THIAM Amadou Moctar et KNORST Nicolas }
\date{Janvier 2022}

\newpage
\begin{document}

\maketitle

\section{Efficacité environnement de travail}

\subsection{ Exercice 1 : }

\begin{itemize}
    \item \textbf{Désactivation de notre souris} : \\
    \\ \# \textit{\textbf{xinput --list}} → permet de lister les périphériques connectés ; \\
    \# \textit{\textbf{xinput -set-prop "id" "Device Enabled" 0}}→ pour désactiver la souris en remplaçant \textit{\textbf{id}} par la valeur affichée.\\
    
\item \textbf{Tableau de problèmes et de ses correctifs :} \\

\end{itemize} 
\begin{tabular}{|c|lr|}
     \hline 
     Problème & Correctif\\
     \hline
     Ouvri le terminal sous Ubuntu & Ctrl + Alt + T\\
     \hline
     Démarrer Firefox ou Brave depuis le terminal & \$ firefox ou brave-browser \\
     \hline
     Saisir directement sur la barre d'URL & Ctrl + E \\
     \hline
     Page précédente / suivante & Alt + ← | Alt + → \\
     \hline
     Switcher entre les onglets du terminal (raccourcis créer) & Alt + ← | Alt + → \\
\end{tabular}

\subsection{Exercice 2 :}

\begin{itemize}
    \item Nous avons choisi TypingClub comme site web pour amélioré notre saisie automatique ;
    \item Nous l'avons choisi parce que : 
    \begin{itemize}
        \item Une interface utilisateur belle et propre ;
        \item Possibilité de s'entrainer pas à pas pour progresser à son rythme et avoir un aperçu de ses progrès ;
        \item Possibilité de créer un compte et de sauvegarder ses ses progrès ;
        \item Possibilité d'apprendre en fonction de la disposition de son clavier ;
        \item Possibilité d'apprendre en mode jeu. 
        \item Capture de l'interface
    \end{itemize}
    
\end{itemize}

\begin{center}
	\includegraphics[scale=0.30]{40} \\
	\hfill \break
	\hfill \break
	\includegraphics[scale=0.30]{41}
\end{center}

\subsection{Exercice 3 :}
	\begin{itemize}
		\item Effectuer les tutoriels avec VIM :
	\end{itemize}
	\begin{center}
		\includegraphics[scale=0.30]{42}
	\end{center}
	\begin{itemize}
		\item Paramétrage de GNU Readline :
			\begin{itemize}
				\item \textit{\textbf{\$ set -o vi}}
			\end{itemize}
		\item L'éditeur choisi est Vim
		\item  Vim comme éditeur par défaut : dans le fichier \textit{.bashrc} : 
			\begin{itemize}
			\item \textit{\textbf{export VISUAL=vim}}
			\item \textit{\textbf{export EDITOR="\$VISUAL"}}
		\end{itemize}
	\end{itemize}
	
\subsection{Exercice 4 :}
	\begin{itemize}
		\item Regarder notre historique :
		\begin{itemize}
			\item \textit{\textbf{\$ history}} ou bien \textit{\textbf{\$ cat /home/user/.bash\_history}}
		\end{itemize}
		\item Oui il y en certaines commandes sensibles et la façon de gérer cela est d'empêcher ces commandes de l'historique bash lui-même
		\item Pour éviter que certaines commandes de l'historique n'apparaissent, dans le fichier \textit{.bashrc}, on ajoute la ligne suivante :
		\begin{itemize}
			\item \textit{HISTIGNORE = ls cd pwd}
		\end{itemize}
		
	\end{itemize}

\subsection{Exercice 5 :}
	\begin{itemize}
		\item Écriture d'une bash fonction mkcd : \\ Dans le fichier \textit{.bashrc}, on ajoute les lignes suivantes : 
		
		\begin{itemize}
			\item \textit{mkcd() \{ \\ mkdir "\$1" ; \\ cd "\$1" ; \\ \}}
		\end{itemize}
		
		\item Écriture de la fonction gitemergency :
			\begin{itemize}
				\item \textit{gitemergency() \{ \\
                git add .; \\
                git commit -m "@"; \\
                git push origin Head; \\
            \}}
			\end{itemize}
	\end{itemize}
	
\subsection{Exercice 6 : }

	\begin{itemize}
		\item \textbf{\textit{\$ touch backup.sh }}→ création du fichier backup
		\item \textbf{\textit{\$ chmod +x backup.sh}} 
		\item Code du fichier \textit{bash\_completion} : 
	\end{itemize}
	\begin{tabular}{|l|}
		\hline
			\hfill \break \\
			if ! shopt -oq posix; then \\
  			if [ -f /usr/share/bash-completion/bash\_completion ]; then \\
    				. /usr/share/bash-completion/bash\_completion \\
  			elif [ -f /etc/bash\_completion ]; then \\
    				. /etc/bash\_completion \\
  			fi \\
			fi \\
			\hfill		
		
	\end{tabular}
	\begin{itemize}
		\item \textbf{\textit{\$ source bash\_autocompletion}}
	\end{itemize}
	

\subsection{Exercice 7 : }
	\begin{itemize}
		\item Installation de ZSH :
			\begin{itemize}
				\item \textbf{\textit{(sudo) apt install zsh -y }}
			\end{itemize}
		\item Modification du prompt pour inclure les informations de vagrant : \\
		Les configurations sont à faire dans le fichier \textit{.zshrc}, en spécifiant : 
		\begin{itemize}
		 \item \textit{ZSH\_THEME=agnoster} → pour spécifier le nom du thème choisi ;
		 \item \textit{plugin=vagrant-prompt.plugin.zsh} → choisir le nom du 
		 \item Dans le fichier \textit{.oh-my-zsh/themes/agnoster.zsh-theme}, on ajoute les lignes suivantes : 
		\end{itemize}
	\end{itemize}
	PROMPT='\%\{\$fg[\$NCOLOR]\%\}\%B\%n\%b\%\{\$reset\_color\%\}:\%\{\$fg[blue]\%\}\%B\%c/\%b\%\{\$reset\_color\%\} \$(vagrant\_prompt\_info)\$(svn\_prompt\_info)\$(git\_prompt\_info)\%(!.\#.\$)' → modification de variable de prompt ;
	
	\hfill \break
	- Ajout des variables de couleurs : \\
	\includegraphics[scale=0.5]{44}
	

\subsection*{Exercice 9 :}
	\begin{itemize}
		\item Installation de trois émulateurs de terminaux à savoir guake, roxterm et xterm: 
			\begin{itemize}
				\item \textit{\textbf{\$ sudo apt-get install -y guake roxterm xterm}}
			\end{itemize}
		\item Nous avons choisi Guake comme émulateur parce qu'il est : léger, rapide, multi-tab et hautement personnalisable.
		\item Suppression des autres : 
		\begin{itemize}
				\item \textit{\textbf{\$ sudo apt-get --purge remove -y roxterm xterm}}
			\end{itemize}
	\end{itemize}
	
\newpage
	\section{SSH :}
	\subsection{Exerice 1 :}
	\begin{itemize}
		\item \textbf{\textit{\$ vagrant up }} →  Démarrer l'environnement Vagrant : 
		\item \textbf{\textit{\$ ssh bob@srv.local }} → connexion avec l'utilisateur bob
		\item \textbf{\textit{\$ ssh alice@srv.local }} → connexion avec l'utilisateur alice
		\item \textbf{\textit{\$ ssh carole@srv.local }} → connexion avec l'utilisateur carole 
		\item Vérification qu'on est sur la Vagrant :
	\end{itemize}
	
	\includegraphics[scale=0.5]{51}
	\includegraphics[scale=0.5]{52}
	\includegraphics[scale=0.5]{53}
	\begin{itemize}
		\item On remarque que la commande de la connexion SSH sur la machine vagrant et présente
	\end{itemize}
	
	


\end{document}
